% !TeX program = lualatex

% setup
\documentclass[
hyperref={bookmarks=false},
xcolor={dvipsnames,svgnames*,x11names*}, 
12pt
]{beamer}
\usepackage{../beamertheme/beamerinnerthemetb}
\usepackage{../beamertheme/beamerouterthemetb}
\usepackage{../beamertheme/beamercolorthemetb}
\usepackage{../beamertheme/beamerthemetb}

% packages
\usepackage{emoji}
\usepackage{graphicx}
\usepackage{xurl}
\usepackage{epigraph}
\usepackage{listings}
\setlength\epigraphwidth{0.7\linewidth}

\definecolor{Rstring}{HTML}{036A07}
\definecolor{Rcomment}{HTML}{4C886B}
\lstset{
	basicstyle=\color{black}\ttfamily\scriptsize,
	backgroundcolor=\color{white}, 
	breaklines=true, 
	keepspaces=true, 
	escapechar={|},
	breakindent=0pt,
	linewidth=1.05\linewidth
}

% options
\setlength{\leftmargini}{0cm}
\hypersetup{
	pdfauthor={Andrea Gilardi},
	colorlinks=true,
	urlcolor=Blue
}

%% Define new stuff for slide numbers
\setbeamertemplate{navigation symbols}{}

\pgfkeys{/visual counter/.cd,
	thickness/.store in=\thickness,
	thickness=0.4ex,
	radius/.store in=\radius,
	radius=1.5ex,
	segment distance/.store in=\segdist,
	segment distance=8,
	color current frame/.store in=\colcurrframe,
	color current frame=orange,
	color old frame/.store in=\cololdframe,
	color old frame=blue,
	color next frame/.store in=\colnextframe,
	color next frame=gray!30,
	color page number/.store in=\colpagenum,
	color page number=white,
	current value/.store in=\currentv,
	current value=1,
	total value/.store in=\totalv,
	total value=2,
	circled page number/.code={
		\begin{tikzpicture}[fill color/.style={}]
			\pgfkeys{/visual counter/.cd, 
				current value=\insertframenumber,
				total value=\inserttotalframenumber,
			}
			\pgfmathtruncatemacro\current{\currentv+1}
			\def\tot{\totalv}
			\def\radiusout{\radius}
			\def\radiusin{\radius-\thickness}
			
			\foreach \s in {1,...,\tot}
			{
				\ifnum\s>\current%
				\tikzset{fill color/.append style={\colnextframe}}%
				\fi%
				\ifnum\s=\current%
				\tikzset{fill color/.append style={\colcurrframe}}%
				\fi%
				\ifnum\s<\current%
				\tikzset{fill color/.append style={\cololdframe}}%
				\fi%
				\fill[fill color]
				({90-360/\tot * (\s - 1)-\segdist}:\radiusout) arc 
				({90-360/\tot * (\s - 1)-\segdist}:{90-360/\tot * (\s)+\segdist}:\radiusout) --
				({90-360/\tot * (\s)+\segdist}:\radiusin) arc 
				({90-360/\tot * (\s)+\segdist}:{90-360/\tot * (\s - 1)-\segdist}:\radiusin);
				% new addition
				\node[inner sep=0pt,text=\colpagenum] at (0,0){\insertframenumber};
			}
		\end{tikzpicture}
	},
}

\setbeamertemplate{footline}{
	\begin{beamercolorbox}[wd=0.95\textwidth, ht=2ex,dp=1ex,sep=1ex]{footline}
		\hfill%
		\tikz\node[/visual counter/.cd,
		segment distance=-2pt,
		radius=0.33cm, thickness=0.33cm,
		color old frame=black,
		color current frame=black!80!gray!50,
		color next frame=black!80!gray!50,
		circled page number,
		]{};
	\end{beamercolorbox}
}
%% End stuff for slide numbers

% metadata
\title{R4DS - Unit 4: R Packages}
\author{Andrea Gilardi}
\date{\today}

\begin{document}
\inserttitlepage

\begin{frame}{Outline and main concepts}
\vspace{-0.5cm}
\begin{itemize}
\itemsep 3ex
\item The objective of this unit is to present the basic structure of an R package and a set of simple but powerful routines that can be used to build our first package. 
\item I will showcase the relevant tools and, at the end, we will build an R package that can be used to decide the optimal date for an happy hour. 
\item Finally, we will synch our package with Github. 
\end{itemize}
\end{frame}	

\begin{frame}{What is an R package?}
\vspace{-0.5cm}
\begin{itemize}
\itemsep 3ex
\item Following the \href{https://r-pkgs.org/}{\emph{R Package - 2e}} book, we could say that \emph{An \textbf{R package} is the fundamental unit of shareable R code. A package bundles together code, data, documentation, and tests, and is easy to share with others.} 
\item An R package can be stored on CRAN (Comprehensive R Archive Network), whereas its development version can be stored on Github or other hosting services 
\item R packages are organised in a standardised format that we must follow. Organising code always makes your life easier since we can follow a template. 
\end{itemize}
\end{frame}	

\begin{frame}{What is an R package? (cont)}
\vspace{-0.5cm}
\begin{itemize}
\itemsep 2ex
\item A bit of terminology now\dots\,A \textbf{package} is a directory of files which extend R containing, at minimum, the files DESCRIPTION and NAMESPACE and an R/ directory. 
\item \textbf{A package is not a library}. 
\item Beware, maintaining and updating an R package can be an extremely time consuming process\dots 
\end{itemize}
\vspace{-0.25cm}
\epigraph{Maybe I'll become a theoretician. Nobody expects you to maintain a theorem.}{Doug Bates (about keeping both Matrix and RcppEigen in sync with CHOLMOD) lme4-author (September 2013)}
\end{frame}

\begin{frame}[fragile]{Peek at the desired product}
\vspace{-0.5cm}
\begin{itemize}
\itemsep 2ex
\item Now we are going to develop an R package named \texttt{statsAndBooze}. The objective of this package is to find the optimal date for happy hour given a set of constraints. 
\item So, for example, 
\begin{lstlisting}
|\textcolor{blue}{library}|(statsAndBooze)
beer_dates <- parse_dates(
  dates = list(
    andrea = |\textcolor{Rstring}{"2023-03-27 / 2023-04-01"}| |\textcolor{Rcomment}{\# available from 27/3 to 1/4}|
    federico = c(|\textcolor{Rstring}{"2023-03-29"}|, |\textcolor{Rstring}{"2023-04-02"}|) |\textcolor{Rcomment}{\# available on 2 days}|
  )
)
decide_happy_hour(beer_dates)
[1] "2023-03-29"
\end{lstlisting}
\end{itemize}
\end{frame}

\begin{frame}[fragile]{Let's start from scratches...}
\vspace{-0.5cm}
\begin{itemize}
\itemsep 2ex
\item During this class we are going to use the R package \texttt{devtools}, so please take a moment to check if it is already installed and, if necessary, install it. 
\item The following code can be used to generate the skeleton of an empty R package named \texttt{packageName}:
\begin{lstlisting}
|\textcolor{blue}{library}|(devtools)
create_package(|\textcolor{Rstring}{"path/for/the/R/packageName"}|)
\end{lstlisting}
So, for example, I'm going to run
\begin{lstlisting}
|\textcolor{blue}{library}|(devtools)
create_package(|\textcolor{Rstring}{"D:/git/statsAndBooze"}|)
\end{lstlisting}
\item The chosen path should point to a non-existing directory that will be created by Rstudio. Do not store an R package inside another R package or a Git repo. 
\end{itemize}
\end{frame}

\begin{frame}[fragile]{Let's start from scratches\dots\;(cont)}
\vspace{-0.5cm}
\begin{itemize}
\itemsep 1ex
\item The previous command should open a new Rstudio session that contains the skeleton of an empty R package. We will explore its content in a couple of minutes.  
\item You should also see a log message like
\begin{lstlisting}
v Creating 'D:/git/statsAndBooze/'
v Setting active project to 'D:/git/statsAndBooze'
v Creating 'R/'
v Writing 'DESCRIPTION'
Package: statsAndBooze
Title: What the Package Does (One Line, Title Case)
Version: 0.0.0.9000
Authors@R (parsed):
* First Last <first.last@example.com> [aut, cre] (YOUR-ORCID-ID)

[truncated...]
\end{lstlisting}
Now we can analyse the log more precisely together! 
\end{itemize}
\end{frame}

\begin{frame}{Let's start from scratches\dots\, (cont)}
\vspace{-0.5cm}
The following lists the content of the new directory: 
\begin{itemize}
\item \texttt{.gitignore}: The file used to control Git versioning; 
\item \texttt{.Rbuildignore}: Similarly to the \texttt{.gitignore} file, this file can be used to exclude some files from the package BUILD. 
\item \texttt{DESCRIPTION}: Stores the metadata of your package (e.g. author, description, dependencies, \dots)
\item \texttt{R/}: The directory where we put the R scripts. 
\item \texttt{NAMESPACE}: declares the functions your package exports and the external functions your package imports from other packages. \textbf{DO NOT EDIT BY HAND}. 
\end{itemize}
Check \href{https://r-pkgs.org/whole-game.html\#create_package}{here} for more details. 
\end{frame}

\begin{frame}[fragile]{Add Git for Version Control}
\vspace{-0.5cm}
\begin{itemize}
\itemsep 2ex
\item Now we started working in a new R session. Therefore, unless you are sharing some suspicious files (\emoji{face-with-symbols-on-mouth}) between sessions, we need to reload \texttt{devtools}.
\item Then, the \texttt{use\_git()} function can be used to initialise a Git project inside the repository. 
\item If you are running in an interactive session, the software might ask you to perform the first commit and restart Rstudio. You should accept both offers. 
\item After the restart, there is a Git panel in Rstudio! 
\end{itemize}
\end{frame}

\begin{frame}{Edit the \texttt{DESCRIPTION}}
\vspace{-0.5cm}
Now we can edit the \texttt{DESCRIPTION} file:
\begin{itemize}
\itemsep 1ex
\item \texttt{Title:} \texttt{Find the best day to have a beer!};
\item \texttt{Version:} I refer you to \url{https://semver.org/}; 
\item \texttt{Authors@R}: Just add name, surname, email, and ORCID; 
\item \texttt{License}: I refer you to \url{https://choosealicense.com/};
\item \texttt{Description:} We are going to fill it at the end.
\end{itemize}
In case you are developing an R package with some co-authors, you need to list their names (and roles) in the Authors field. See \href{https://r-pkgs.org/description.html\#sec-description-authors-at-r}{here} for more details. 
\end{frame}

\begin{frame}[fragile]{Package dependencies}
\vspace{-0.5cm}
\begin{itemize}
\itemsep 2ex
\item As already mentioned, we are trying to develop an R package that parses dates and intervals. 
\item Working with dates is a \href{https://lubridate.tidyverse.org/articles/lubridate.html\#if-anyone-drove-a-time-machine-they-would-crash}{nightmare}\dots\, Therefore, instead of defining our own routines, we are going to create wrappers to another package: \texttt{lubridate}!
\item Please notice that when you develop an R package, you cannot use \texttt{\textcolor{blue}{library}()} since that works only for interactive scripting. See also \href{https://r-pkgs.org/dependencies-mindset-background.html}{here} and \href{https://r-pkgs.org/dependencies-in-practice.html}{here} for more details.  
\item Instead, we can use the function \texttt{use\_package(\textcolor{Rstring}{"pkg"})}. 
\end{itemize}
\end{frame}

\begin{frame}[fragile]{Package dependencies}
\vspace{-0.5cm}
\begin{itemize}
\itemsep 2ex
\item You should now see the following output: 
\begin{lstlisting}
> use_package(|\textcolor{Rstring}{"lubridate"}|)
v Adding 'lubridate' to Imports field in DESCRIPTION
* Refer to functions with `lubridate::fun()`
\end{lstlisting}
The message highlights that whenever we refer to a lubridate function, we need to add the \texttt{lubridate::} prefix. 

\item Check also the DESCRIPTION file and see what happened. 
\item Please notice that the same behaviour must be applied to any function which is not included into the \textbf{base} package. 
\item \textbf{Question:} How can you determine which package defines a function? 
\end{itemize}
\end{frame}

\begin{frame}[fragile]{Interactive development}
\vspace{-0.5cm}
\begin{itemize}
\itemsep 2ex
\item Usually, it is much easier if you run the first tests into an interactive session before writing into the package. 
\item Our first objective is to define a function which takes in input a list of strings and returns the parsed dates:
\begin{lstlisting}
parse_dates(
  dates = list(
    andrea = |\textcolor{Rstring}{"2023-03-29"}|, |\textcolor{Rcomment}{\# exactly with this format}|
    federico = |\textcolor{Rstring}{"2023-03-30"}| |\textcolor{Rcomment}{\# exactly with this format}|
  )
)
$andrea
[1] "2023-03-29" |\textcolor{Rcomment}{\# NB: it must have class = "Date"}|

$federico
[1] "2023-03-30"
\end{lstlisting}
\item Now it's your turn! Try to code such a function. 
\end{itemize}
\end{frame}

\begin{frame}[fragile]{The first function}
\vspace{-0.5cm}
\begin{itemize}
\itemsep 2ex
\item Now that we have sketched the skeleton of the function, we can add it to our package. First, we need to create an R script into the R/ folder. We can run \texttt{use\_R(\textcolor{Rstring}{"path.R"})}. 
\item So, for example, I'm going to run 
\begin{lstlisting}
use_R(|\textcolor{Rstring}{"parse.R"}|)
v Setting active project to 'D:/git/statsAndBooze'
* Modify 'R/parse.R'
* Call `use_test()` to create a matching test file
\end{lstlisting}
\item We copy the function definition (\textbf{and only the function definition}) into the new script. Every time that we refer to a lubridate function, we need to add \texttt{lubridate::}. 
\end{itemize}
\end{frame}

\begin{frame}[fragile]{The first function (cont)}
\vspace{-0.5cm}
\begin{itemize}
\itemsep 2ex
\item The \texttt{parse.R} file might look like 
\begin{lstlisting}
1. parse_dates <- |\textcolor{blue}{function}|(x) {
2.   lapply(x, lubridate::as_date)
3. }
4. 
\end{lstlisting}
\item Now we can save the file and restart the R session. Then, if you want to make \texttt{parse\_dates()} available for testing, you can restart the R session and run \texttt{load\_all()}. 
\item Let's try it together\dots \;NB: delete the function definition and any superfluous library call from the script used for interactive testing. 
\item If everything works right, now it's a good time for a commit. 
\end{itemize}
\end{frame}

\begin{frame}[fragile]{R CMD check}
\vspace{-0.5cm}
\begin{itemize}
\itemsep 3ex
\item Every time that you modify your R package in a non-negligible way (e.g. you add a new function), you should test that all its moving parts are still working. 
\item The \texttt{R CMD check} command is the gold standard for checking R packages. 
\item We can run it from the Build panel or via the \texttt{check()} function (which is also defined in \texttt{devtools}). 
\item Let's try!
\end{itemize}
\end{frame}

\begin{frame}[fragile]{Documentation}
\vspace{-0.5cm}
\begin{itemize}
\itemsep 2ex 
\item Unfortunately, our new function doesn't have an help file:
\begin{lstlisting}
> devtools::load_all(".")
i Loading statsAndBooze
> ?parse_dates
No documentation for ‘parse_dates’ in specified packages and libraries: you could try ‘??parse_dates’
\end{lstlisting}
\item We can write a specially formatted comment right above the function definition to generate its help page via an R package named \texttt{roxygen2}. 
\item From Rstudio, open \texttt{parse.R}, place your cursor somewhere into the function definition and then click on Code -> Insert Roxygen Skeleton. 
\end{itemize}
\end{frame}

\begin{frame}[fragile]{Documentation (cont)}
\vspace{-0.5cm}
\begin{itemize}
\itemsep 2ex
\item You should see something like
\begin{lstlisting}
|\textcolor{Rcomment}{\#' Title}|
|\textcolor{Rcomment}{\#'}|
|\textcolor{Rcomment}{\#'}| |\textcolor{blue}{@param}| x 
|\textcolor{Rcomment}{\#'}|
|\textcolor{Rcomment}{\#'}| |\textcolor{blue}{@return}|
|\textcolor{Rcomment}{\#'}| |\textcolor{blue}{@export}|
|\textcolor{Rcomment}{\#'}|
|\textcolor{Rcomment}{\#'}| |\textcolor{blue}{@examples}|
parse_dates <- function(x) {
	lapply(x, lubridate::as_date)
}
\end{lstlisting}
\item Now we are going to fill all the relevant parts. 
\end{itemize}
\end{frame}

\begin{frame}[fragile]{Documentation (cont)}
\vspace{-0.5cm}
\begin{itemize}
\itemsep 2ex
\item At the end, the output should look like
\begin{lstlisting}
|\textcolor{Rcomment}{\#' Parse a list of strings into dates}|
|\textcolor{Rcomment}{\#'}|
|\textcolor{Rcomment}{\#'}| |\textcolor{blue}{@details}| |\textcolor{Rcomment}{Please notice that each date must be specified using the YYYY-MM-DD format.}|
|\textcolor{Rcomment}{\#'}|
|\textcolor{Rcomment}{\#'}| |\textcolor{blue}{@param}| dates |\textcolor{Rcomment}{A list of strings specifying dates.}|
|\textcolor{Rcomment}{\#'}|
|\textcolor{Rcomment}{\#'}| |\textcolor{blue}{@return}| |\textcolor{Rcomment}{A list with the same length as the input. The strings are converted into objects of class Date.}|
|\textcolor{Rcomment}{\#'}| |\textcolor{blue}{@export}|
|\textcolor{Rcomment}{\#'}|
|\textcolor{Rcomment}{\#'}| |\textcolor{blue}{@examples}|
|\textcolor{Rcomment}{\#' 1 + 1}|
parse_dates <- function(dates) {
  lapply(dates, lubridate::as_date)
}
\end{lstlisting}
\end{itemize}
\end{frame}

\begin{frame}[fragile]{Documentation (cont)}
\vspace{-0.5cm}
\begin{itemize}
\itemsep 3ex
\item We can run \texttt{document()} to let \texttt{roxygen2} do its magic. 
\item If you explore the NAMESPACE you should now see 
\begin{lstlisting}
|\textcolor{Rcomment}{\# Generated by roxygen2: do not edit by hand}|

export(parse_dates)
\end{lstlisting}
\item Let's run the R CMD check again. 
\item If everything goes right, this is a good time for another commit! 
\end{itemize}
\end{frame}

\begin{frame}[fragile]{A minimal R package}
\vspace{-0.5cm}
\begin{itemize}
\itemsep 2ex
\item Now we have a minimal working package! We can install it by running \texttt{devtools::install()} or using the Build panel. 
\item After installing our package, we can run the following in a fresh R session
\begin{lstlisting}
|\textcolor{blue}{library}|(statsAndBooze)
beer_dates <- list(
  |\textcolor{Rcomment}{\# We can see that our function works with >= 2 people and >= 2 dates}|
  andrea = c(|\textcolor{Rstring}{"2023-03-29"}|, |\textcolor{Rstring}{"2023-03-30"}|), 
  federico = |\textcolor{Rstring}{"2023-03-30"}|,
  chiara = |\textcolor{Rstring}{"2023-03-30"}|
)
parse_dates(beer_dates)
\end{lstlisting}
\item If you don't see any error, that means our super simple package works \emoji{tada}
\end{itemize}
\end{frame}

\begin{frame}[fragile]{To infinity and beyond \emoji{rocket}}
\vspace{-0.5cm}
\begin{itemize}
\itemsep 3ex
\item Now it's time to expand our package! In fact, we said that our objective is to decide a common day for an happy hour and, currently, we are not doing that\dots 
\item In fact, for the moment we are just parsing the input constraints into a list of \texttt{Date} objects, but we are missing the key step: the organization of the happy hour!  
\item As for the previous case, it's really convenient to start running the first tests into an interactive session. 
\end{itemize}
\end{frame}

\begin{frame}[fragile]{\texttt{decide\_happy\_hour()} function}
\vspace{-0.5cm}
\begin{itemize}
\item How would you programmatically determine the common day in the following list?
\begin{lstlisting}
|\textcolor{blue}{library}|(statsAndBooze)
list_dates <- list(
  andrea = c(|\textcolor{Rstring}{"2023-03-29"}|, |\textcolor{Rstring}{"2023-03-30"}|), 
  federico = |\textcolor{Rstring}{"2023-03-30"}|,
  chiara = |\textcolor{Rstring}{"2023-03-30"}|
)
parsed_dates <- parse_dates(list_dates)
decide_happy_hour <- |\textcolor{blue}{function}|(x) {
	... 
}
\end{lstlisting}
\item We need to recursively apply the same function to check which availabilities are shared among different people\dots\; See the next slide for a possible solution :)
\end{itemize}
\end{frame}

\begin{frame}[fragile]{\texttt{decide\_happy\_hour()} function}
\vspace{-0.5cm}
\begin{itemize}
\itemsep 2ex
\item My suggestion would be something like
\begin{lstlisting}
decide_happy_hour <- |\textcolor{blue}{function}|(x) {
  lubridate::as_date(Reduce(lubridate::intersect, x))
}
\end{lstlisting}
\item First, let's see if it works in an interactive session. 
\item If you don't see any problem, create an ad-hoc \texttt{.R} file (e.g. \texttt{decide.R}), add the new function, and document it. 
\item \textbf{Exercise for home:} Try to understand why we do need the extra call to \texttt{lubridate::as\_date}.
\end{itemize}
\end{frame}

\begin{frame}[fragile]{\texttt{decide\_happy\_hour()} function}
\vspace{-0.5cm}
\begin{itemize}
\itemsep 2ex
\item Now, after completing all the previous steps we can load\footnote{\textbf{NB:} load $\ne$ install. See \texttt{?devtools::load\_all} for more details.} our package and, in a fresh R session, retest that everything works properly: 
\begin{lstlisting}
> devtools::load_all(|\textcolor{Rstring}{"."}|)
i Loading statsAndBooze
> list_dates <- list(
    andrea = |\textcolor{Rstring}{"2023-03-30"}|, 
    federico = |\textcolor{Rstring}{"2023-03-30"}|
  )
> parsed_dates <- parse_dates(list_dates)
> decide_happy_hour(parsed_dates)
[1] "2023-03-30"
\end{lstlisting}
\item If everything looks right, rerun R CMD check. Please notice that R CMD check re-documents our package. 
\end{itemize}
\end{frame}

\begin{frame}[fragile]{\texttt{decide\_happy\_hour()} function}
\vspace{-0.5cm}
\begin{itemize}
\itemsep 2ex
\item \textbf{Question:} What is the expected output of the following code? Please notice that there is no common date.  
\begin{lstlisting}
> devtools::load_all(|\textcolor{Rstring}{"."}|)
i Loading statsAndBooze
> list_dates <- list(
    andrea = |\textcolor{Rstring}{"2023-03-29"}|, 
    federico = |\textcolor{Rstring}{"2023-03-30"}|
  )
> parsed_dates <- parse_dates(list_dates)
> decide_happy_hour(parsed_dates)
\end{lstlisting}
Try to formulate an hypothesis and test it running the code. 
\item Finally, if you don't see any problem, commit again! 
\end{itemize}
\end{frame}

\begin{frame}[fragile]{Reinstall and more docs}
\vspace{-0.5cm}
\begin{itemize}
\itemsep 2ex
\item Now it is a good time to reinstall our R package and check again that everything works as expected. 
\item You should see following
\begin{lstlisting}
|\textcolor{blue}{library}|(statsAndBooze)
list_dates <- list(
  andrea = c(|\textcolor{Rstring}{"2023-04-01"}|, |\textcolor{Rstring}{"2023-04-02"}|), 
  federico = c(|\textcolor{Rstring}{"2023-04-02"}|, |\textcolor{Rstring}{"2023-04-03"}|), 
  chiara = |\textcolor{Rstring}{"2023-04-02"}|
)
parsed_dates <- parse_dates(list_dates)
decide_happy_hour(parsed_dates)
[1] 2023-04-02
\end{lstlisting}
\item We can also finish the docs by filling in some examples and complete the DESCRIPTION. Then CHECK and commit!
\end{itemize}
\end{frame}

\begin{frame}[fragile]{Unit testing}
\vspace{-0.5cm}
\begin{itemize}
\itemsep 2ex
\item The example reported in the previous slide informally shows that our R package works in a particular case. 
\item Now we want to formalise our expectations into \textbf{unit tests}!
\item Why do we need unit testing? Two main reasons: 
\begin{enumerate}
\itemsep 1ex
\item We might want to test what happens with wrong inputs or other edge cases that can remain hidden for the end users;
\item We want to ensure that all aspects of our package keep working even after refactoring its main functionalities. 
\end{enumerate}
\end{itemize}
\end{frame}

\begin{frame}[fragile]{Unit testing (cont)}
\vspace{-0.5cm}
\begin{itemize}
\itemsep 1ex
\item After loading \texttt{devtools}, you can run \texttt{use\_testthat()} to setup the unit testing environment. 
\begin{lstlisting}
> use_testthat()
v Setting active project to 'D:/git/statsAndBooze'
v Adding 'testthat' to Suggests field in DESCRIPTION
v Setting Config/testthat/edition field in DESCRIPTION to '3'
v Creating 'tests/testthat/'
v Writing 'tests/testthat.R'
* Call `use_test()` to initialize a basic test file and open it for editing.
\end{lstlisting}
\item Then, we can run \texttt{use\_test(\textcolor{Rstring}{<file>})} to create a new test file. So, for example, I can run \texttt{use\_test(\textcolor{Rstring}{"parse"})}. 
\begin{lstlisting}
> use_test("parse")
v Writing 'tests/testthat/test-parse.R'
* Modify 'tests/testthat/test-parse.R'
\end{lstlisting}
\end{itemize}
\end{frame}

\begin{frame}[fragile]{Unit testing (cont)}
\vspace{-0.5cm}
\begin{itemize}
\itemsep 2ex
\item Now we need to edit the newly created file and write our unit test(s). First, we need to provide a short sentence that summarises the objective of the test. For example
\begin{lstlisting}
test_that(|\textcolor{Rstring}{"parse\_dates(): basic functionalities work"}|, {
	expect_equal(2 * 2, 4)
})
\end{lstlisting}
\item The tests are actually run using the R package \texttt{testthat} which exports several helper functions (\texttt{expect\_length()}, \texttt{expect\_message()}, \texttt{expect\_error()}, \dots) to test different aspects of our package (equality, differences, \dots).
\item Then, we need to write the corpus of the test comparing observed output and our expectation. 
\end{itemize}
\end{frame}

\begin{frame}[fragile]{Unit testing (cont)}
\vspace{-0.5cm}
\begin{itemize}
\itemsep 2ex
\item For example: 
\begin{lstlisting}
test_that(|\textcolor{Rstring}{"parse\_dates(): basic functionalities work"}|, {
  input_strings <- list(
    andrea = |\textcolor{Rstring}{"2023-04-03"}|,
    marco = |\textcolor{Rstring}{"2023-04-03"}|
  )
  expected_dates <- list(
    andrea = lubridate::as_date(|\textcolor{Rstring}{"2023-04-03"}|),
    marco = lubridate::as_date(|\textcolor{Rstring}{"2023-04-03"}|)
  )
  expect_equal(parse_dates(input_strings), expected_dates)
})
\end{lstlisting}
\item After loading the package (\texttt{load\_all()}), we can run the new test interactively as any other R function. 
\end{itemize}
\end{frame}

\begin{frame}[fragile]{Unit testing (cont)}
\vspace{-0.5cm}
\begin{itemize}
\itemsep 2ex
\item The same procedure can be repeated for the other function
\begin{lstlisting}
> use_test(|\textcolor{Rstring}{"decide"}|)
v Setting active project to 'D:/git/statsAndBooze'
v Writing 'tests/testthat/test-decide.R'
* Modify 'tests/testthat/test-decide.R'
\end{lstlisting}
\item The actual test might look like
\begin{lstlisting}
test_that(|\textcolor{Rstring}{"decide\_happy\_hour(): basic functionalities work"}|, {
  beer_dates <- list(
    andrea = lubridate::as_date(|\textcolor{Rstring}{"2023-04-03"}|),
    federico = lubridate::as_date(|\textcolor{Rstring}{"2023-04-03"}|),
    chiara = lubridate::as_date(|\textcolor{Rstring}{"2023-04-03"}|)
  )
  expect_equal(
    decide_happy_hour(beer_dates), 
    lubridate::as_date(|\textcolor{Rstring}{"2023-04-03"}|)
  )
})
\end{lstlisting}
\end{itemize}
\end{frame}

\begin{frame}[fragile]{Unit testing (cont)}
\vspace{-0.5cm}
\begin{itemize}
\itemsep 1ex
\item You should also test some pathological cases which might not be directly exposed to the regular end users
\begin{lstlisting}
test_that(|\textcolor{Rstring}{"decide\_happy\_hour(): empty intersection"}|, {
  beer_dates <- list(
    andrea = lubridate::as_date(|\textcolor{Rstring}{"2023-04-03"}|),
    marco = lubridate::as_date(|\textcolor{Rstring}{"2023-04-04"}|)
  )
  expect_equal(
    decide_happy_hour(beer_dates),
    lubridate::as_date(numeric(|\textcolor{blue}{0}|))
  )
})
\end{lstlisting}
\item Similarly, you could develop a test to control the behaviour of the function in case of misspecified inputs. 
\item For example: what is the expected output when one or more of the input Date(s) is NA?
\end{itemize}
\end{frame}

\lstset{escapechar={£}}

\begin{frame}[fragile]{Unit testing (cont)}
\vspace{-0.5cm}
\begin{itemize}
\itemsep 2ex
\item The function \texttt{test()} (which is also defined in \texttt{devtools}) automatically runs all tests in a package and returns an informative output (with a funny comment on the results)
\begin{lstlisting}
> test()
i Testing statsAndBooze
v | F W S  OK | Context
v |         2 | decide
v |         1 | parse

== Results ===============================
Duration: 0.5 s

[ FAIL 0 | WARN 0 | SKIP 0 | £\textcolor{Rcomment}{PASS}£ 3 ] 

£\textcolor{Rcomment}{You are a coding rockstar!}£
\end{lstlisting}
\item The same behaviour occurs when you run R CMD check. Let's try and if everything looks right, we can also commit.  
\end{itemize}
\end{frame}

\begin{frame}[fragile]{Github!}
\vspace{-0.5cm}
\begin{itemize}
\itemsep 2ex
\item The \texttt{use\_github()} function can be used to automatically setup a Github project starting from our package. 
\item It runs the following steps (and many more, see the help page): 
\begin{enumerate}
\item Checks the initial state of the repo; 
\item Creates an associated repo on Github; 
\item Configures the appropriate remote so you can automatically run \texttt{push}/\texttt{pull} commands. 
\end{enumerate}
\item \textbf{You need to be authenticated via a working PAT}.
\item Let's test it together so we can check the messages! 
\end{itemize}
\end{frame}

\begin{frame}[fragile]{Synch with a remote}
\vspace{-0.5cm}
\begin{itemize}
\itemsep 1ex
\item The \texttt{git pull}\footnote{\texttt{git pull} is a combination of \texttt{git fetch} (download data) and \texttt{git merge} (merge data). See \href{https://git-scm.com/book/en/v2/Git-Basics-Working-with-Remotes}{here} for more details.} command (or the blue arrow pointing downward in Rstudio) can be used to automatically synchronise your local work starting from a remote (the \texttt{origin} remote, by default). 
\item Now, if you click that arrow in Rstudio, the software should open a new panel with the following text
\begin{lstlisting}
£\textcolor{blue}{>>> C:/Program Files/Git/bin/git.exe pull}£
Already up to date.
\end{lstlisting}
\itemsep 0ex
\item If you modify some files and you want to share your changes with others, you have to push them upstream. Let's try it together using the Rstudio GUI. 
\end{itemize}
\end{frame}

\begin{frame}[fragile]{Synch with a remote (cont)}
\vspace{-0.5cm}
\begin{itemize}
\itemsep 2ex
\item But what happens if the upstream has some changes you don't have in your local version? Your push will be rejected!  
\item For example, let's modify the DESCRIPTION file on Github and commit the change. Then, let's add one toy example to \texttt{parse\_dates()} and run \texttt{document()}. 
\item If you commit and push, you should see the following
\begin{lstlisting}
£\textcolor{blue}{>>> git.exe push origin HEAD:refs/heads/main}£
To https://github.com/agila5/statsAndBooze.git
! [rejected]    HEAD -> main (fetch first)
error: failed to push some refs to [truncated...]
hint: Updates were rejected because the remote contains work that you do not have locally. This is usually caused by another repository pushing to the same ref. You may want to first integrate the remote changes (e.g., 'git pull ...') before pushing again. See the 'Note about fast-forwards' in 'git push --help' for details.
\end{lstlisting}
\end{itemize}
\end{frame}

\begin{frame}[fragile]{Synch with a remote (cont)}
\vspace{-0.5cm}
\begin{itemize}
\itemsep 1ex
\item In the best case scenario, Git can harmonise the two versions! 
\item Therefore, following the hint, we can \texttt{pull} from upstream 
\begin{lstlisting}
£\textcolor{blue}{>>> C:/Program Files/Git/bin/git.exe pull}£
From https://github.com/agila5/statsAndBooze
de846da..a79986f  main       -> origin/main
Merge made by the 'ort' strategy.
DESCRIPTION | 2 +-
1 file changed, 1 insertion(+), 1 deletion(-)
\end{lstlisting}
(check the message!) and then you repeat the push 
\begin{lstlisting}
£\textcolor{blue}{>>> git.exe push origin HEAD:refs/heads/main}£
To https://github.com/agila5/statsAndBooze.git
a79986f..2086e99  HEAD -> main
\end{lstlisting}
\end{itemize}
\end{frame}

\begin{frame}[fragile]{Synch with a remote (cont)}
\vspace{-0.5cm}
\begin{itemize}
\itemsep 1ex
\item But what can you do when your commit and the upstream are in two states that Git cannot automatically merge? 
\item For example, what happens if we replace the Description in DESCRIPTION both on Github and in our locale? 
\item If we commit and push the changes, we get the same rejection message as before. 
\item However, if we pull from our remote we see the following: 
\begin{lstlisting}
£\textcolor{blue}{>>> C:/Program Files/Git/bin/git.exe pull}£
From https://github.com/agila5/statsAndBooze
2086e99..39b642e  main       -> origin/main
Auto-merging DESCRIPTION
CONFLICT (content): Merge conflict in DESCRIPTION
Automatic merge failed; fix conflicts and then commit the result
\end{lstlisting}
\end{itemize}
\end{frame}

\begin{frame}[fragile]{Synch with a remote (cont)}
\vspace{-0.5cm}
\begin{itemize}
\itemsep 0ex
\item The DESCRIPTION file should look like
\begin{lstlisting}
£\textcolor{blue}{Package}£: statsAndBooze
£\textcolor{blue}{Title}£: Find the best day to have a beer!
£\textcolor{blue}{Version}£: 0.0.0.9000
£\textcolor{blue}{Authors@R}£: [omitted...]
<<<<<<< HEAD
£\textcolor{blue}{Description}£: DEF
=======
£\textcolor{blue}{Description}£: ABC
>>>>>>> 39b642e405fd0bdd0efef9fb79b182dab278e66d
\end{lstlisting}
\item As we can see, Git adds conflict-resolution markers.
\itemsep 1ex
\item The upstream version of the file (i.e. what is saved on Github) is summarised between \texttt{<<<<<} and \texttt{======}, while your local version lies in the bottom part. 
\item Using any text editor you can manually fix the conflict. Then \texttt{add} the files and repeat the commit plus push process. 
\end{itemize}
\end{frame}

\begin{frame}{I want to work with my friends!}
\vspace{-0.5cm}
\begin{itemize}
\itemsep 3ex
\item If you want to add collaborators to your (public or private) Github repo, you can execute the following steps:
\begin{enumerate}
\item Go to the Github page of your repo and click Settings; 
\item Then, click on Collaborators in the section named Access; 
\item Finally, click on the green button named Add people and write the name or the ID of the person you want to invite.  
\end{enumerate}
\item Let's try it together! Each group should nominate a team leader that must add all the other team members (plus me!) as collaborators to his/her repository. 
\item Please note that you can customise the permissions of your collaborators. More details \href{https://docs.github.com/en/organizations/managing-user-access-to-your-organizations-repositories/repository-roles-for-an-organization}{here}. 
\end{itemize}
\end{frame}

\begin{frame}[fragile]{I want to work with my friends!}
\vspace{-0.5cm}
\begin{itemize}
\itemsep 2ex
\item Clearly, when several people work on the same Github repo, it's really easy to create a merge conflict. Let's try! 
\item Each group can simulate the process: 
\begin{itemize}
\item First, one of you should modify a file and commit the changes to Github. The process should work out smoothly. 
\item Then, another one modifies the same part of the same file and commit the changes. If you push now, Github should reject your changes. 
\item If the second guy pulls the new changes, he/she will get a merge conflict. Fix it together and push the new version of that file! Finally, repeat the process changing the roles. 
\end{itemize}
\end{itemize}
\end{frame}

\begin{frame}{Git branching (\emoji{crying-face})}
\vspace{-0.5cm}
\begin{itemize}
\itemsep 2ex
\item Following the Git Book, we can say that \emph{``Branching means you diverge from the main line of development and continue to do work without messing with that main line"}.
\item This is a quite convenient feature, especially when several people work on the same project.
\item Unfortunately, we cannot cover this topic here, but I refer you to the \href{https://git-scm.com/book/en/v2/Git-Branching-Branches-in-a-Nutshell}{manual} for more details. 
\item If you want, we can also organise a quick (1h) extra class after the Easter holidays to . 
\end{itemize}
\end{frame}

\begin{frame}
\vspace{2cm}
\begin{center}
\Huge
\textbf{THE END!}
\end{center}
\vspace{1.5cm}
\epigraph{E quindi uscimmo a riveder le stelle}{Dante Alighieri, Inferno XXXIV, 139}
\end{frame}
\end{document}