% !TeX program = lualatex

% setup
\documentclass[
hyperref={bookmarks=false},
xcolor={dvipsnames,svgnames*,x11names*}, 
12pt
]{beamer}
\usepackage{../beamertheme/beamerinnerthemetb}
\usepackage{../beamertheme/beamerouterthemetb}
\usepackage{../beamertheme/beamercolorthemetb}
\usepackage{../beamertheme/beamerthemetb}

% packages
\usepackage{emoji}
\usepackage{graphicx}
\usepackage{xurl}
\usepackage{epigraph}
\usepackage{listings}
\setlength\epigraphwidth{\linewidth}

\definecolor{Rstring}{HTML}{036A07}
\definecolor{Rcomment}{HTML}{4C886B}
\lstset{
	basicstyle=\color{black}\ttfamily\scriptsize,
	backgroundcolor=\color{white}, 
	breaklines=true, 
	keepspaces=true, 
	escapechar={|},
	breakindent=0pt,
	linewidth=1.05\linewidth
}

% options
\setlength{\leftmargini}{0cm}
\hypersetup{
	pdfauthor={Andrea Gilardi},
	colorlinks=true,
	urlcolor=Blue
}

%%% Define new stuff for slide numbers
\setbeamertemplate{navigation symbols}{}

\pgfkeys{/visual counter/.cd,
	thickness/.store in=\thickness,
	thickness=0.4ex,
	radius/.store in=\radius,
	radius=1.5ex,
	segment distance/.store in=\segdist,
	segment distance=8,
	color current frame/.store in=\colcurrframe,
	color current frame=orange,
	color old frame/.store in=\cololdframe,
	color old frame=blue,
	color next frame/.store in=\colnextframe,
	color next frame=gray!30,
	color page number/.store in=\colpagenum,
	color page number=white,
	current value/.store in=\currentv,
	current value=1,
	total value/.store in=\totalv,
	total value=2,
	circled page number/.code={
		\begin{tikzpicture}[fill color/.style={}]
			\pgfkeys{/visual counter/.cd, 
				current value=\insertframenumber,
				total value=\inserttotalframenumber,
			}
			\pgfmathtruncatemacro\current{\currentv+1}
			\def\tot{\totalv}
			\def\radiusout{\radius}
			\def\radiusin{\radius-\thickness}
			
			\foreach \s in {1,...,\tot}
			{
				\ifnum\s>\current%
				\tikzset{fill color/.append style={\colnextframe}}%
				\fi%
				\ifnum\s=\current%
				\tikzset{fill color/.append style={\colcurrframe}}%
				\fi%
				\ifnum\s<\current%
				\tikzset{fill color/.append style={\cololdframe}}%
				\fi%
				\fill[fill color]
				({90-360/\tot * (\s - 1)-\segdist}:\radiusout) arc 
				({90-360/\tot * (\s - 1)-\segdist}:{90-360/\tot * (\s)+\segdist}:\radiusout) --
				({90-360/\tot * (\s)+\segdist}:\radiusin) arc 
				({90-360/\tot * (\s)+\segdist}:{90-360/\tot * (\s - 1)-\segdist}:\radiusin);
				% new addition
				\node[inner sep=0pt,text=\colpagenum] at (0,0){\insertframenumber};
			}
		\end{tikzpicture}
	},
}

\setbeamertemplate{footline}{
	\begin{beamercolorbox}[wd=0.95\textwidth, ht=2ex,dp=1ex,sep=1ex]{footline}
		\hfill%
		\tikz\node[/visual counter/.cd,
		segment distance=-2pt,
		radius=0.33cm, thickness=0.33cm,
		color old frame=black,
		color current frame=black!80!gray!50,
		color next frame=black!80!gray!50,
		circled page number,
		]{};
	\end{beamercolorbox}
}
%%% End stuff for slide numbers

% metadata
\title{R4DS - Unit 4: R Packages}
\author{Andrea Gilardi}
\date{\today}

\begin{document}
\inserttitlepage

\begin{frame}{Outline and main concepts}
\vspace{-0.5cm}
\begin{itemize}
\itemsep 3ex
\item The objective of this unit is to present the basic structure of an R package and a set of simple but powerful routines that can be used to build our first package. 
\item I will showcase the relevant tools and, at the end, we will build an R package that can be used to decide the optimal date for an happy hour. 
\item Finally, we will synch our package on Github. 
\end{itemize}
\end{frame}	

\begin{frame}{What is an R package?}
\vspace{-0.5cm}
\begin{itemize}
\itemsep 3ex
\item Following the \href{https://r-pkgs.org/}{\emph{R Package - 2e}} book, we could say that \emph{An \textbf{R package} is the fundamental unit of shareable R code. A package bundles together code, data, documentation, and tests, and is easy to share with others.} 
\item An R package can be stored on CRAN (Comprehensive R Archive Network), whereas its development version can be stored on Github or other hosting services 
\item R packages are organised in a standardised format that we must follow. Organising code always makes your life easier since we can follow a template. 
\end{itemize}
\end{frame}	

\begin{frame}{What is an R package? (cont)}
\vspace{-0.5cm}
\begin{itemize}
\itemsep 2ex
\item A bit of terminology now\dots\,A \textbf{package} is a directory of files which extend R containing, at minimum, the files DESCRIPTION and NAMESPACE and an R/ directory. 
\item \textbf{A package is not a library}. 
\item Beware, maintaining and updating an R package can be an extremely time consuming process\dots 
\end{itemize}
\vspace{-0.25cm}
\epigraph{Maybe I'll become a theoretician. Nobody expects you to maintain a theorem.}{Doug Bates (about keeping both Matrix and RcppEigen in sync with CHOLMOD) lme4-author (September 2013)}
\end{frame}

\begin{frame}[fragile]{Peek at the desired product}
\vspace{-0.5cm}
\begin{itemize}
\itemsep 2ex
\item Now we are going to develop an R package named \texttt{statsAndBooze}. The objective of this package is to find the optimal date for an happy hour given a set of constraint. 
\item So, for example, 
\begin{lstlisting}
|\textcolor{blue}{library}|(statsAndBooze)
beer_dates <- parse_dates(
  dates = list(
    andrea = |\textcolor{Rstring}{"27-03-2023/1-4-2023"}| |\textcolor{Rcomment}{\# available from 27/3 to 1/4}|
    federico = c(|\textcolor{Rstring}{"29-03-2023"}|, |\textcolor{Rstring}{"2-4-2023"}|) |\textcolor{Rcomment}{\# available on 2 days}|
  )
)
decide_happy_hour(beer_dates)
[1] "2023-03-29"
\end{lstlisting}
\end{itemize}
\end{frame}

\begin{frame}[fragile]{Let's start from scratches...}
\vspace{-0.5cm}
\begin{itemize}
\itemsep 2ex
\item During this we class we are going to use the R package \texttt{devtools}, so please take a moment to check if it is already installed and, if necessary, install it. 
\item The following code can be used to generate the skeleton of an empty R package:
\begin{lstlisting}
|\textcolor{blue}{library}|(devtools)
create_package(|\textcolor{Rstring}{"path/for/the/R/package"}|)
\end{lstlisting}
So, for example, in a few minutes I'm going to run
\begin{lstlisting}
|\textcolor{blue}{library}|(devtools)
create_package(|\textcolor{Rstring}{"D:/git/statsAndBooze"}|)
\end{lstlisting}
\item The chosen path should point to a non-existing directory that will be created by Rstudio. Do not store an R package inside another R package or a Git repo. 
\end{itemize}
\end{frame}

\begin{frame}[fragile]{Let's start from scratches\dots\;(cont)}
\vspace{-0.5cm}
\begin{itemize}
\itemsep 1ex
\item The previous command should open a new Rstudio session that contains the skeleton of an empty R package. We will explore its content in a couple of minutes.  
\item You should also see a log message like
\begin{lstlisting}
v Creating 'D:/git/statsAndBooze/'
v Setting active project to 'D:/git/statsAndBooze'
v Creating 'R/'
v Writing 'DESCRIPTION'
Package: statsAndBooze
Title: What the Package Does (One Line, Title Case)
Version: 0.0.0.9000
Authors@R (parsed):
* First Last <first.last@example.com> [aut, cre] (YOUR-ORCID-ID)

[truncated...]
\end{lstlisting}
Now we can analyse the output more precisely together! 
\end{itemize}
\end{frame}

\begin{frame}{Let's start from scratches\dots\, (cont)}
\vspace{-0.5cm}
The following lists the content of the new directory: 
\begin{itemize}
\item \texttt{.gitignore}: The file used to control Git versioning; 
\item \texttt{.Rbuildignore}: Similarly to the \texttt{.gitignore} file, this file can be used to exclude some files from the package BUILD. 
\item \texttt{DESCRIPTION}: Stores the metadata of your package (e.g. author, description, dependencies, \dots)
\item \texttt{R/}: The directory where we put the R scripts. 
\item \texttt{NAMESPACE}: declares the functions your package exports and the external functions your package imports from other packages. \textbf{DO NOT EDIT BY HAND}. 
\end{itemize}
Check \href{https://r-pkgs.org/whole-game.html\#create_package}{here} for more details. 
\end{frame}

\begin{frame}[fragile]{Add Git for Version Control}
\vspace{-0.5cm}
\begin{itemize}
\itemsep 2ex
\item Now we started working in a new R session. Therefore, unless you are sharing some \texttt{.Rdata} files (\emoji{face-with-symbols-on-mouth}) between sessions, we need to reload \texttt{devtools}.
\item Then, the \texttt{use\_git()} function can be used to initialise a Git project inside the repository. 
\item If you are running in an interactive session, the software might ask you do perform the first commit and restart Rstudio. You should accept the offer. 
\item As we can see, now there is a Git panel in Rstudio. 
\end{itemize}
\end{frame}

\begin{frame}{Edit the \texttt{DESCRIPTION}}
\vspace{-0.5cm}
Now we can edit the \texttt{DESCRIPTION} file:
\begin{itemize}
\itemsep 1ex
\item \texttt{Title:} \texttt{Find the best day to have a beer!};
\item \texttt{Version:} I refer you to \url{https://semver.org/}; 
\item \texttt{Authors@R}: Just add name, surname, email, and ORCID; 
\item \texttt{License}: I refer you to \url{https://choosealicense.com/};
\item \texttt{Description:} We are going to fill it at the end.
\end{itemize}
In case you are developing an R package with some co-authors, you need to list their names (and roles) in the Authors field. See \href{https://r-pkgs.org/description.html\#sec-description-authors-at-r}{here} for more details. 
\end{frame}

\begin{frame}[fragile]{Package dependencies}
\vspace{-0.5cm}
\begin{itemize}
\itemsep 2ex
\item As already mentioned, we are trying to develop an R package that parses dates and intervals. 
\item Working with dates is a nightmare so, instead of defining our own routines, we are going to create wrappers to another package, namely \texttt{lubridate}!
\item Please notice that when you develop an R package, you cannot use \texttt{\textcolor{blue}{library}()} since that works only for interactive scripting. See also \href{https://r-pkgs.org/dependencies-mindset-background.html}{here} and \href{https://r-pkgs.org/dependencies-in-practice.html}{here} for more details.  
\item Instead, we can use the function \texttt{use\_package(\textcolor{Rstring}{"pkg"})}. 
\end{itemize}
\end{frame}

\begin{frame}[fragile]{Package dependencies}
\vspace{-0.5cm}
\begin{itemize}
\itemsep 2ex
\item You should now see the following output: 
\begin{lstlisting}
> use_package(|\textcolor{Rstring}{"lubridate"}|)
v Adding 'lubridate' to Imports field in DESCRIPTION
* Refer to functions with `lubridate::fun()`
\end{lstlisting}
The message highlights that whenever we refer to a lubridate function, we need to add the \texttt{lubridate::} prefix. Check also the DESCRIPTION file. 
\item Please notice that the same behaviour must be applied to any function which is not included into the \textbf{base} package. 
\item \textbf{Question:} How can you see which package defines a function? 
\end{itemize}
\end{frame}

\begin{frame}[fragile]{Interactive development}
\vspace{-0.5cm}
\begin{itemize}
\itemsep 2ex
\item Usually, it is much easier if you run the first tests into an interactive session before writing into the package. 
\item Our first objective is to define a function which takes in input a list of strings and returns the parsed dates:
\begin{lstlisting}
parse_dates(
  dates = list(
    andrea = |\textcolor{Rstring}{"2023-03-29"}|, |\textcolor{Rcomment}{\# exactly with this format}|
    federico = |\textcolor{Rstring}{"2023-03-30"}| |\textcolor{Rcomment}{\# exactly with this format}|
  )
)
$andrea
[1] "2023-03-29" |\textcolor{Rcomment}{\# NB: it must have class = "Date"}|

$federico
[1] "2023-03-30"
\end{lstlisting}
\item Now it's your turn! Try to code such a function. 
\end{itemize}
\end{frame}

\begin{frame}[fragile]{The first function}
\vspace{-0.5cm}
\begin{itemize}
\itemsep 2ex
\item Now that we have sketched the skeleton of the function, we can add it to our package. First, we need to create an R script into the R/ folder. We can run \texttt{use\_R(\textcolor{Rstring}{"path.R"})}. 
\item So, for example, I'm going to run 
\begin{lstlisting}
use_R(|\textcolor{Rstring}{"parse.R"}|)
v Setting active project to 'D:/git/statsAndBooze'
* Modify 'R/parse.R'
* Call `use_test()` to create a matching test file
\end{lstlisting}
\item We copy the function definition (\textbf{and only the function definition}) into the new script. Every time that we refer to a lubridate function, we need to add \texttt{lubridate::}. 
\end{itemize}
\end{frame}

\begin{frame}[fragile]{The first function (cont)}
\vspace{-0.5cm}
\begin{itemize}
\itemsep 2ex
\item The \texttt{parse.R} file might look like 
\begin{lstlisting}
1. parse_dates <- |\textcolor{blue}{function}|(dates) {
2.   lapply(dates, lubridate::as_date)
3. }
4. 
\end{lstlisting}
\item Now we can save the file and restart the R session. Then, if you want to make \texttt{parse\_dates()} available for testing, you can restart the R session and run \texttt{load\_all()}. 
\item Let's try it together\dots \;NB: delete the function definition and any superfluous library call from the script used for interactive testing. 
\item If everything works right, now it's a good time for a commit. 
\end{itemize}
\end{frame}

\begin{frame}[fragile]{R CMD check}
\vspace{-0.5cm}
\begin{itemize}
\itemsep 3ex
\item Every time that you modify your R package in a non-negligible way (e.g. you add a new function), you should test that all its moving parts are still working. 
\item The \texttt{R CMD check} command is the gold standard for checking R packages. 
\item We can run it from the Build panel or via the \texttt{check()} function (which is also defined in \texttt{devtools}). 
\item Let's try!
\end{itemize}
\end{frame}

\begin{frame}[fragile]{Documentation}
\vspace{-0.5cm}
\begin{itemize}
\itemsep 2ex 
\item Unfortunately, our new function doesn't have an help file:
\begin{lstlisting}
> devtools::load_all(".")
i Loading statsAndBooze
> ?parse_dates
No documentation for ‘parse_dates’ in specified packages and libraries: you could try ‘??parse_dates’
\end{lstlisting}
\item We can write a specially formatted comment right above the function definition to generate its help page via an R package named \texttt{roxygen2}. 
\item From Rstudio, open \texttt{parse.R}, place your cursor somewhere into the function definition and then click on Code -> Insert Roxygen Skeleton. 
\end{itemize}
\end{frame}

\begin{frame}[fragile]{Documentation (cont)}
\vspace{-0.5cm}
\begin{itemize}
\itemsep 2ex
\item You should see something like
\begin{lstlisting}
|\textcolor{Rcomment}{\#' Title}|
|\textcolor{Rcomment}{\#'}|
|\textcolor{Rcomment}{\#'}| |\textcolor{blue}{@param}| dates 
|\textcolor{Rcomment}{\#'}|
|\textcolor{Rcomment}{\#'}| |\textcolor{blue}{@return}|
|\textcolor{Rcomment}{\#'}| |\textcolor{blue}{@export}|
|\textcolor{Rcomment}{\#'}|
|\textcolor{Rcomment}{\#'}| |\textcolor{blue}{@examples}|
parse_dates <- function(dates) {
	lapply(dates, lubridate::as_date)
}
\end{lstlisting}
\item Now we are going to fill all the relevant parts. 
\end{itemize}
\end{frame}

\begin{frame}[fragile]{Documentation (cont)}
\vspace{-0.5cm}
\begin{itemize}
\itemsep 2ex
\item At the end, the output should look like
\begin{lstlisting}
|\textcolor{Rcomment}{\#' Parse a list of strings into dates}|
|\textcolor{Rcomment}{\#'}|
|\textcolor{Rcomment}{\#' Please notice that each date must be specified using the YYYY-MM-DD format.}|
|\textcolor{Rcomment}{\#'}|
|\textcolor{Rcomment}{\#'}| |\textcolor{blue}{@param}| dates |\textcolor{Rcomment}{A list of strings specifying dates. Each entry must be related to a different person.}|
|\textcolor{Rcomment}{\#'}|
|\textcolor{Rcomment}{\#'}| |\textcolor{blue}{@return}| |\textcolor{Rcomment}{A list with the same length as the input. The strings are converted into objects of class Date.}|
|\textcolor{Rcomment}{\#'}| |\textcolor{blue}{@export}|
|\textcolor{Rcomment}{\#'}|
|\textcolor{Rcomment}{\#'}| |\textcolor{blue}{@examples}|
|\textcolor{Rcomment}{\#' 1 + 1}|
parse_dates <- function(dates) {
  lapply(dates, lubridate::as_date)
}
\end{lstlisting}
\end{itemize}
\end{frame}

\begin{frame}[fragile]{Documentation (cont)}
\vspace{-0.5cm}
\begin{itemize}
\itemsep 3ex
\item We can run \texttt{document()} to let \texttt{roxygen2} do its magic. 
\item If you explore the NAMESPACE you should see 
\begin{lstlisting}
|\textcolor{Rcomment}{\# Generated by roxygen2: do not edit by hand}|

export(parse_dates)
\end{lstlisting}
\item Let's run the R CMD check again. 
\item If everything goes right, this is a good time for another commit! 
\end{itemize}
\end{frame}

\begin{frame}[fragile]{A minimal R package}
\vspace{-0.5cm}
\begin{itemize}
\itemsep 2ex
\item Now we have a minimal working package! We can install it by running \texttt{devtools::install()} or using the Build panel. 
\item After installing our package, we can run the following in a fresh R session
\begin{lstlisting}
|\textcolor{blue}{library}|(statsAndBooze)
beer_dates <- list(
  |\textcolor{Rcomment}{\# We can see that our function works with n >= 2 people}|
  andrea = c(|\textcolor{Rstring}{"2023-03-29"}|, |\textcolor{Rstring}{"2023-03-30"}|), 
  federico = |\textcolor{Rstring}{"2023-03-30"}|,
  chiara = |\textcolor{Rstring}{"2023-03-30"}|
)
parse_dates(beer_dates)
\end{lstlisting}
\item If you don't see any error, that means our super simple package works \emoji{tada}
\end{itemize}
\end{frame}

\begin{frame}[fragile]{To infinity and beyond \emoji{rocket}}
\vspace{-0.5cm}
\begin{itemize}
\itemsep 3ex
\item Now it's time to expand our package! In fact, we said that our objective is to decide a common day for an happy hour. 
\item For the moment, we just parsed the input constraints into a list of \texttt{Date} objects but we are missing the key step: the organization of the happy hour!  
\item As for the previous case, it's really convenient to start running the first tests into an interactive session. 
\end{itemize}
\end{frame}

\begin{frame}[fragile]{\texttt{decide\_happy\_hour()} function}
\vspace{-0.5cm}
\begin{itemize}
\item How would you programmatically determine the common day in the following list?
\begin{lstlisting}
|\textcolor{blue}{library}|(statsAndBooze)
beer_dates <- list(
  andrea = c(|\textcolor{Rstring}{"2023-03-29"}|, |\textcolor{Rstring}{"2023-03-30"}|), 
  federico = |\textcolor{Rstring}{"2023-03-30"}|,
  chiara = |\textcolor{Rstring}{"2023-03-30"}|
)
beer_dates <- parse_dates(beer_dates)
find_happy_hour <- |\textcolor{blue}{function}|(beer_dates) {
	... 
}
\end{lstlisting}
\item We need to recursively apply the same function to check which days are shared among pairs of inputs\dots\; See the next slide for a solution :)
\end{itemize}
\end{frame}

\begin{frame}[fragile]{\texttt{decide\_happy\_hour()} function}
\vspace{-0.5cm}
\begin{itemize}
\itemsep 2ex
\item My suggestion would be something like
\begin{lstlisting}
find_happy_hour <- |\textcolor{blue}{function}|(dates) {
  lubridate::as_date(Reduce(lubridate::intersect, dates))
}
\end{lstlisting}
\item First, let's see if it works in an interactive session. Then, we can create an ad-hoc \texttt{.R} file (e.g. \texttt{decide.R}), add the new function and document it. 
\item \textbf{Exercise for home:} Try to understand why we do need the extra call to \texttt{lubridate::as\_date}.
\end{itemize}
\end{frame}

\begin{frame}[fragile]{\texttt{decide\_happy\_hour()} function}
\vspace{-0.5cm}
\begin{itemize}
\itemsep 2ex
\item Now, after completing all the previous steps we can load\footnote{\textbf{NB:} load $\ne$ install. See \texttt{?devtools::load\_all} for more details.} our package and, in a fresh R session, retest that everything works properly: 
\begin{lstlisting}
> devtools::load_all(|\textcolor{Rstring}{"."}|)
i Loading statsAndBooze
> beer_dates <- list(
    andrea = |\textcolor{Rstring}{"2023-03-30"}|, 
    federico = |\textcolor{Rstring}{"2023-03-30"}|
  )
> beer_dates <- parse_dates(beer_dates)
> decide_happy_hour(beer_dates)
[1] "2023-03-30"
\end{lstlisting}
\item If everything looks right, let's rerun R CMD check. Please notice that R CMD check automatically re-document our package. 
\end{itemize}
\end{frame}

\begin{frame}[fragile]{\texttt{decide\_happy\_hour()} function}
\vspace{-0.5cm}
\begin{itemize}
\itemsep 2ex
\item \textbf{Question:} What is the expected output for the following input? 
\begin{lstlisting}
list(
  andrea = |\textcolor{Rstring}{"2023-03-29"}|, 
  federico = |\textcolor{Rstring}{"2023-03-30"}|
)
\end{lstlisting}
Try to formulate an hypothesis and test it running the code. 
\item Finally, if you don't see any problem, commit again! 
\end{itemize}
\end{frame}

\begin{frame}[fragile]{Missing steps!}
\vspace{-0.5cm}
As you can imagine, we just created a super simple package that can be enhanced in many ways. For example: 
\begin{itemize}
\itemsep 2ex
\item \texttt{use\_readme\_rmd()} is a simple routine that can be used to add a README file that documents the main functionalities of our package. 
\item \texttt{use\_github()} is another helper function that we can use to link our local Git repository to an online Github session. Please notice that you must not have a Github directory with the same name as the current package. 
\item \textbf{Testing!} We can formulate a series of unit tests to ensure that the functionalities in our package are preserved even if we modified the R code. During the next class we will start from here. 
\end{itemize}
\end{frame}

\begin{frame}[fragile]{Homework!}
TODO :) I will update the slides with the homework this evening. 
\end{frame}

\end{document}